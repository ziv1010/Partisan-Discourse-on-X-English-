\section{Discussion}
We study political discourse of Indian influencers in this paper to understand if they exhibit polarization with respect to various social and political issues. In this direction, we first develop a proxy for their political leaning using the number of political retweets they receive around their tweets. Next, we analyze the stance of their tweets around different aspects, using an SLM fine-tuned on their tweets. Our findings conclusively prove that most Indian influencers are significantly partisan when it comes to their tweets on various aspects. The aggregate partisanism also is dichotomous in nature, wherein there exists two distinct communities of influencers, one in favor of the ruling dispensation and the other against it.

There are several nationally and regionally prominent political parties functioning in India. However, the dichotomy that we observe in the influencer discourse provides concrete indication of the highly bipolar nature of the Indian polity. We see that influencers with a significant political leaning towards the ruling dispensation disproportionately favor issues like \textit{Hindutva, Hinduism, Kashmiri Pandits} and write against \textit{farmer's protests, muslims, bharatjodoyatra}. On the other hand, the pro-opposition influencers follow the exact opposite trend around the same issues. We thus see a clear trend of influencers kotowing their favorite party lines. With the recent trend of increasing religious and political polarization in India [ref], this especially does not augur well. Influencers are highly popular figures, many of whom are revered and closely followed on social media by numerous users. A bipolar nature of the influencer discourse thus amplifies the extant biases in the society, thereby leading to a fragmented democracy. 

We also see that these trends are uniformly observed for both English and Hindi tweets, indicative of the fact that the observed polarization holds across languages. In a country where a significant fraction of the social media population consumes tweets in Hindi [ref], this is a sign of a polarized political discourse cutting across linguistic borders.

The topical analysis of tweets reveals ...

This work comes as a timely intervention around influencer polarization in India, provided that several recent studies have targeted similar research questions. Our work acts as a formative step towards large-scale analysis of influencer discourse on social media (from 2020-2023). A major contribution of this study is the development of a generalizable research framework to study aspect based stance on social media. While we study influencer discourse in India corresponding to two languages, the framework is applicable to any geography and can be extended to work on other languages as well. The use of an SLM to perform stance analysis also highlights the need to incorporate low-infrastructure methods to perform large-scale data analysis.

As part of future work, we intend to extend the study to other regional languages in India, to see if similar trends are observed. Furthermore, while the current work focuses on an overall analysis of influencer discourse, a temporal study capturing the evolution of influencer leaning and discourse partisanism can be undertaken. The current study only considers influencer tweets that have received at least one political retweet (retweet by a politician). However, it would be interesting to include other tweets on similar aspects, followed by a stance analysis. A comparative analysis of the trends can then provide us with stronger empirical evidence of partisanism in tweets, irrespective of the political endorsements received by them.

\section{Conclusion}
In this paper, we examined the political discourse of Indian influencers to assess the extent and nature of polarization across a range of social and political issues. To this end, we first proposed a proxy to infer influencers’ political leanings based on the political retweets received in response to their content. We then analyzed the stances expressed in their tweets with respect to different issue-specific aspects, leveraging a small language model fine-tuned on influencer-generated data. Our analysis provides strong empirical evidence that a majority of Indian influencers exhibit pronounced partisanship in their engagement with these issues. Furthermore, polarization at the aggregate level is distinctly dichotomous, revealing the presence of two well-defined communities of influencers—one broadly aligned with the ruling dispensation and the other positioned in opposition. These findings highlight the influential role of content creators in shaping and reinforcing polarized political discourse on social media platforms.
