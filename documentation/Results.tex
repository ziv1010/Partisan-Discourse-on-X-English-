\section{Results}

This section reports the empirical outcomes of the stance classification pipeline applied to Indian political influencer discourse on X. The analysis covers 680,847 tweet--keyword pairs across 38 political keywords, with influencers categorized by political affiliation into Pro-Ruling (462 influencers) and Pro-Opposition (498 influencers) groups.%% ============================================================
%% SECTION 5.1: MODEL PERFORMANCE
%% ============================================================
\subsection{Model Performance}

To validate our choice of the LoRA-adapted Mistral-7B model, we compared its performance against the baseline architectures described in the methodology. Figure~\ref{fig:model_comparison} illustrates the comparative results across key metrics. The LoRA-tuned Mistral model consistently outperforms the baseline approaches, demonstrating superior capability in capturing the nuances of political stance.

\begin{figure}[H]
  \centering
  \includegraphics[width=0.95\linewidth]{../final_visualisations/model_comparison.png}
  \caption{Performance comparison of the fine-tuned Mistral (LoRA) model against baseline architectures.}
  \label{fig:model_comparison}
\end{figure}

The confusion matrix presented in Figure~\ref{fig:confusion_matrix} further elucidates the model's classification accuracy. We observe a strong diagonal, indicating high precision in correctly identifying Favor, Against, and Neutral stances. Notably, the model effectively minimizes confusion between opposing stances (Favor vs. Against), which is critical for accurate polarization analysis. The instances of misclassification are primarily concentrated between the Neutral category and polar stances, reflecting the inherent ambiguity often present in less explicit political expression.

\begin{figure}[H]
  \centering
  \includegraphics[width=0.95\linewidth]{../final_visualisations/mistral_finetuned_confusion_matrix.png}
  \caption{Confusion matrix for the fine-tuned Mistral model on the held-out test set.}
  \label{fig:confusion_matrix}
\end{figure}

%% ============================================================
%% SECTION 5.2: MULTILINGUAL STANCE TRENDS
%% ============================================================
\subsection{Multilingual Stance Patterns}

We extended our analysis to investigate whether political discourse patterns diverge across languages. Figure~\ref{fig:multilingual_stack} presents a stacked stance distribution for English and Hindi tweets. A striking trend emerges: the stance distribution patterns are remarkably similar across both languages. This observation suggests that the polarization dynamics and narrative structures deployed by political influencers are consistent, irrespective of the linguistic medium. We further corroborated this finding by cross-referencing with a negativity scatter analysis (Figure~\ref{fig:negativity_scatter}), which similarly indicated high congruence in the sentiment and negativity levels associated with specific keywords across both languages.

\begin{figure}[H]
  \centering
  \includegraphics[width=0.95\linewidth]{../final_visualisations/combined_hindi_english_stance_stacked.png}
  \caption{Stacked stance distribution comparison between English and Hindi tweets.}
  \label{fig:multilingual_stack}
\end{figure}

The visualization reveals three distinct bands of behavior. The top row highlights keywords where the Pro-Ruling faction exhibits extreme polarization, predominantly expressing strong favor. The middle row represents a "mixed" zone where both political parties show similar stance distributions, indicating areas of shared narrative or lower polarization. The bottom row serves as the counterpart to the first, showcasing keywords where the Pro-Opposition faction displays extreme polarization. Given this high degree of congruence between English and Hindi discourse, we appended the two datasets for the remainder of our analysis to provide a more comprehensive view of the political landscape.

\begin{figure}[H]
  \centering
  \includegraphics[width=0.95\linewidth]{../final_visualisations/6_negativity_scatter.png}
  \caption{Comparison of negativity scores for keywords in English vs. Hindi tweets.}
  \label{fig:negativity_scatter}
\end{figure}

The scatter plot in Figure~\ref{fig:negativity_scatter} plots the average negativity score of each keyword in English (X-axis) against its score in Hindi (Y-axis). We observe a strong positive correlation, with most keywords clustering along the diagonal. This indicates that topics evoking negative sentiment in English discourse tend to trigger similar levels of negativity in Hindi, reinforcing the hypothesis of a unified partisan narrative that transcends language barriers. The visual evidence suggests that the emotional valence of political topics is consistent across linguistic communities, further justifying the aggregation of datasets for subsequent analysis.

%% ============================================================
%% SECTION 5.3: NORMALIZED FAVOR RATES (COMBINED)
%% ============================================================
\subsection{Normalized Favor Rates (Combined Data)}

Using the combined English and Hindi dataset, we re-examined the normalized favor rates for each keyword. Figure~\ref{fig:butterfly_favor} displays these rates using a butterfly chart, with Pro-Ruling favor percentages on the right and Pro-Opposition favor percentages on the left.

\begin{figure}[H]
  \centering
  \includegraphics[width=0.95\linewidth]{../final_visualisations/2_butterfly_favor_normalized.png}
  \caption{Butterfly chart showing normalized favor rates by keyword (combined English \& Hindi data).}
  \label{fig:butterfly_favor}
\end{figure}

The combined analysis reinforces the bimodal nature of the discourse. We observe a set of keywords where the Pro-Ruling side exhibits near-complete dominance in favorable expression, contrasting sharply with keywords where the Pro-Opposition side dominates. Topics located near the center of the chart continue to exhibit more balanced favor distributions, but the overall trend confirms that most political topics in the Indian context are heavily skewed toward one side's narrative. This polarization is even more pronounced when leveraging the larger, combined dataset, validating the robustness of our initial findings.

%% ============================================================
%% SECTION 5.4: KEYWORD-LEVEL STANCE DIVERGENCE
%% ============================================================
\subsection{Keyword-Level Stance Divergence}

Figure~\ref{fig:divergence_scatter} presents the keyword-level stance divergence for the combined corpus. This scatter plot maps each keyword based on the difference in favor rates (X-axis) and against rates (Y-axis) between the two political blocs.

\begin{figure}[H]
  \centering
  \includegraphics[width=0.95\linewidth]{../final_visualisations/divergence_scatter_english_hindi_combined.png}
  \caption{Keyword-level stance divergence scatter plot for the combined English and Hindi dataset.}
  \label{fig:divergence_scatter}
\end{figure}

The combined divergence analysis highlights the distinct "camps" occupied by different topics. We can clearly identify clusters of keywords that drive the most significant wedges between the political factions. Outliers in this plot represent topics that generate unique polarization patterns—for instance, topics where one side is highly favorable while the other is not necessarily hostile but silent, or where both are hostile. The distribution of points further confirms that divergence is not uniform; rather, specific high-volume topics drive the bulk of the polarization, while a "long tail" of issues remains relatively contested or neutral.

%% ============================================================
%% SECTION 5.5: COMPLETE STANCE DISTRIBUTION
%% ============================================================
\subsection{Complete Stance Distribution by Keyword}

Figure~\ref{fig:stance_by_keyword} presents the full stance distribution for each keyword, disaggregated by political affiliation. Each keyword displays proportions of favor, neutral, and against stances for both Pro-Ruling and Pro-Opposition influencers.

\begin{figure}[H]
  \centering
  \includegraphics[width=0.95\linewidth]{../final_visualisations/4_stance_distribution_by_keyword.png}
  \caption{Stance distribution by keyword and political affiliation, showing proportions of favor, neutral, and against stances.}
  \label{fig:stance_by_keyword}
\end{figure}

To organize keywords into meaningful thematic categories, we employed Google's Gemini model. The model was provided only with the list of 38 keywords and prompted to create thematic buckets based on its knowledge of Indian political discourse. Table~\ref{tab:stance_composition} presents the resulting keyword categories.

\vspace{0.5em}
\begin{center}
\footnotesize
\captionof{table}{Keywords Grouped by Thematic Category (Generated using Gemini)}
\label{tab:stance_composition}
\begin{tabular}{p{2.4cm}p{4.1cm}}
\toprule
\textbf{Category} & \textbf{Keywords} \\
\midrule
Favor-dominant (both) & ayodhya, mahotsav, ucc \\
Against-dominant (both) & inflation, unemployment, suicides \\
Split (PR favor, PO against) & modi, ram\_mandir, aatmanirbhar, hindutva, farm\_laws, caa \\
Split (PO favor, PR against) & rahulgandhi, congress, farmers\_protests, shaheen\_bagh, muslim \\
Mixed neutral ($>$20\%) & china, gdp, democracy, minorities \\
\bottomrule
\end{tabular}
\end{center}
\vspace{0.3em}
{\footnotesize \textit{Note: PR = Pro-Ruling, PO = Pro-Opposition}}
\vspace{0.5em}

