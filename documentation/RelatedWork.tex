\section{Related Work}
Social media, lately, has become an indispensable source for news and information consumption \cite{pew_social_media_news_2025, toi_reuters_digital_news_india_2024} for the masses. A significant fraction of users stay updated with current affairs through social media, which leads to its impact on public opinion. Previous works report about this impact in various social and political fronts \cite{levy2021social,park2024exposure}, thereby conclusively proving that social media carries the power to reconfigure the very foundations of democracy. However, increasing polarization in social media has a detrimental impact on the society.  In this section, we discuss related work spanning the two primary components of our proposed framework -- analysis of social media polarization, and stance classification.

\subsection{Social Media Polarization}
While originally envisioned as a tool to broaden the horizons of knowledge and ideas, social media has contributed to an increasingly polarized and fragmented global society \cite{cicchini2022news,kofi_annan_social_media_polarization}. Polarization leads to imbalance in opinion formation, hindering individuals from obtaining a holistic view of salient issues. Numerous examples of the detrimental social impact causes by a polarized social media have been highlighted in prior studies [ref]. With the increasing penetration of digital technologies, policy-makers, think-tanks, and governments are increasingly planning necessary interventions to tackle the undesirable effect social media brings on the society \cite{oecd_facts_not_fakes_2024,carnegie_countering_disinformation_2024}. 

% One of the leading reasons behind social media polarization is the partisanship in the content posted by political entities. Previous studies have highlighted how social media was the key to achieving favorable electoral outcomes in the US presidential elections [ref]. Similar trends have been observed across the globe [ref], including the global south [ref] where increasing partisan content has led to significant polarization of the social media audience. The sources of such content are not just limited to political entities, but extend to traditionally influential figures like actors, athletes, and business-persons [ref]. 

Furthermore, the impact of polarization is often accentuated by \textit{micro-influencers} -- social media creators with dedicated following focused on a specific niche. These micro-influencers (or influencers) usually have a significantly high number of regular followers forming closely knit communities. Prior work has sufficiently exemplified cases where these influencers have promoted certain political agenda through highly partisan content posted through their social media accounts \cite{reinikainen2025convergence}. Owing to their network reach and impact, such content is consumed by their follower base, strengthening the pre-existing echo chambers. Similar trends have also been observed in India \cite{dash2022divided}, which forms the third highest user base for X, capturing around 36\% of all social media interactions globally across platforms \cite{socialsamosa2025influencerreport}.  

Social media polarization has been studied using both content \cite{falkenberg2022growing,nordbrandt2023affective,SUN2023113845} and network analysis methods \cite{esteve2022homophily,candellone2025negative,peralta2024multidimensional,singh2025multi,pournaki2025influencers,zhang2025polarization} in extant literature. %Falkenberg et al. \cite{falkenberg2022growing} study the discussion around the UN Conference of the Parties on Climate Change (COP) using Twitter data, and reveal a large increase in ideological polarization during COP26, contrary to the period between COP20 and COP25.  Overgaard et al. \cite{overgaard2024manifestation} measure affective polarization by looking at the expressed negativity towards certain actors through user comments. Maria Nordbrandt \cite{nordbrandt2023affective} uses Dutch panel data to show that the level of affective polarization affects subsequent use of social media. Sun et al. \cite{SUN2023113845} perform sentiment analysis of 3600 posts on Weibo, and find that content ideology and symbolic expressions are positively related to social media opinion polarization. Munoz et al. \cite{munoz2024quantifying} present a comparative analysis of polarization measurement methods on X during Spanish election cycles, and propose a novel algorithm for capturing polarization in political content.
The connection between partisanship in social media content and polarization has been well established in prior work. Demszky et al. \cite{demszky2019analyzing} use linguistic and semantic content analysis show that polarization is affected mainly by partisan differences in content framing. Flamino et al. \cite{flamino2023political} analyze tweets to examine how partisan news content circulated by media and influencer accounts increased ideological polarization between the 2016 and 2020 US elections. Dash et al. \cite{dash2022divided} quantify how influences engage with politically charged content in a partisan manner, showing that ideological polarization among influencers is linked to increased engagement and retweeting, thus amplifying partisan dynamics.

% Pournaki et al. \cite{pournaki2025influencers} measure political polarization by using topic modeling to identify issues and network analysis (Stochastic Block Models on retweet networks) to cluster users into global left-and right-leaning opinion camps. Zhang et al. \cite{zhang2025polarization} construct user opinion networks, and quantify polarization on social media through a random walk based approach.

Motivated by prior work in this area, we study the political discourse generated by Indian influencers to assess the level of partisanship in it. For this purpose, we first manually annotate a set of influencer generated tweets to create a dataset for aspect based stance classification. It is ensured that these tweets are politically endorsed, i.e., have been retweeted at least once by a political entity. Next, we use this dataset to fine-tune a small language model (Mistral-7B) to develop an aspect based stance classifier. While existing works have focused on various stance classification methods including XXX [ref], XXX [ref], and XXX [ref], we use an SLM based approach to ensure that our framework is generalizable to low resource settings. We also evaluate partisanism in influencer generated content in both English and Hindi, to establish robustness of our findings.
\subsection{Stance Classification}
Stance classification has long been recognized as a challenging NLP task involving the identification of a speaker’s holistic subjective disposition toward a topic or aspect. Prior work has shown that stance is not reducible to sentiment \cite{anand-etal-2011-cats}, and requires leveraging of significantly more nuanced linguistic features. Stance classification methods include both unsupervised \cite{gambini2022tweets2stance}, and supervised \cite{zarrella2016mitre,mohammad2017stance,xu2018cross,kuccuk2018stance} approaches. Prior work has shown that transformer-based approaches often substantially outperform classical approaches for stance detection on social media data \cite{glandt-etal-2021-stance,karande2021stance}. Recent work has additionally examined the use of LLMs for stance classification \cite{ding2024cross}, showing that LLM performance is highly sensitive to prompt design and that parameter-efficient fine-tuning such as LoRA does not consistently outperform zero-shot prompting \cite{cruickshank2023prompting,ma2024chain}. Prior work has also shown that effective stance classification requires rich representations and contextual modeling beyond surface lexical features \cite{hasan-ng-2013-stance}. Our work operationalizes these insights by extending this line of research to large-scale political influencer discourse, integrating LLM-based stance reasoning of multilingual tweet data. Specifically, we test the performance of both basic transformer and LLM based approaches (zero-shot, few-shot, and fine-tuned) in the task of stance classification, on influencer generated political tweets.